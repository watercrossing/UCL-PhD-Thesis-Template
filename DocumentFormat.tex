% A file containing all the custom formatting and packages for this document
\usepackage[utf8]{inputenc}
\usepackage[british]{babel}
\usepackage[autopunct=true]{csquotes}

\usepackage{amsmath}
\usepackage{amssymb}
\usepackage{calc}                   % Used for chapter style
\usepackage[final]{graphicx}
\usepackage[toctitles]{titlesec}    % Put long titles in table of contents and short titles in headers
\usepackage[svgnames,table]{xcolor}       % Use the svgnames list of colour names from the xcolor package
\usepackage{geometry}
\usepackage{eso-pic}


%% Suppress some warning that aren't actually warnings.
\usepackage[saveall]{silence}

%\WarningFilter{biblatex}{'\DeclareSortingScheme' is deprecated}
\WarningFilter{biblatex}{Since you are using the 'memoir' class}
%\WarningFilter{biblatex}{Field 'extrayear' is deprecated}
\WarningFilter{biblatex-chicago}{Empty Ibidem citation}
%\WarningFilter{etex}{Extended allocation already in use}
%\WarningFilter{latex}{destination with the same identifier}

% This warning occurs when using pdf's as images.. sometimes.
\pdfsuppresswarningpagegroup=1


%%%%%%%%%%%%%%%
%%% Colours %%%
%%%%%%%%%%%%%%%

% List of custom colours
\ifthenelse{\boolean{onlinecopy}}{%
\definecolor{Highlight}{RGB}{119, 3, 3}        % Chapter colour
}{\definecolor{Highlight}{RGB}{0, 0, 0}}

\colorlet{CIngolf}{Blue}
\colorlet{CSimon}{Green}

%%%%%%%%%%%%
%% Setup %%%
%%%%%%%%%%%%
\usepackage{hyperref}
\hypersetup{
    pdftitle={\thesisTitle},          % title
    pdfauthor={\thesisAuthor},  % author
    pdfstartview=Fit,           % fit page to pdf reader window
    pdfpagelayout=TwoPageRight,   % show one page, scroll moves to next page
    linkcolor=Highlight,        % colour of internal links
    citecolor=Highlight,        % colour of links to bibliography
    urlcolor=Highlight,         % colour of external links
    colorlinks=true,			% turn coloured links on
    final=true					% use final version of links even in draft mode
}



%%%%%%%%%%%%%%
%%% Styles %%%
%%%%%%%%%%%%%%

% Choose line spacing (otherwise \SingleSpacing or \DoubleSpacing)
\OnehalfSpacing

% List spacing (can also use \firmlists for slightly wider spacing)
\tightlists
\usepackage[inline]{enumitem}
\setlist{noitemsep}

% Sections (make font sizes smaller than usual)
\setsecheadstyle{\large\bfseries\raggedright}
\setsubsecheadstyle{\bfseries\raggedright}
\setsubsubsecheadstyle{\bfseries\raggedright}
\setparaheadstyle{\itshape\raggedright}

% Caption style (small and bold)
\captionnamefont{\bfseries\small}
\captiontitlefont{\small}

% Depth of section labels
\setsecnumdepth{subsubsection}
\settocdepth{subsection}

% Create counter for list of notes, and fancy notes command
\newcounter{notes}
\renewcommand{\thenotes}{\arabic{notes}}
\newcommand{\notes}[3]{%
  \gdef\thereisanote{}%
	\refstepcounter{notes}%
	{\noindent\color{#2}\textbf{#1}: #3}%
	\addcontentsline{lon}{notes}{\protect{\color{#2}\textbf{#1:}} #3}}%\par} %Comment \par back in if notes should have their own paragraph
\newcommand{\listnotesname}{To Do List}
\newlistof{listofnotes}{lon}{\listnotesname}
\newlistentry{notes}{lon}{0}

\AtEndDocument{\ifdefined\thereisanote\label{note:was:used:in:doc}\fi}

% the \@ needs special treating, hence the 'makeat*'
\makeatletter
\newcommand{\conditionalLoN}{\@ifundefined{r@note:was:used:in:doc}{}{\listofnotes}}
\makeatother
%%%%%%%%%%%%
%%% Page %%%
%%%%%%%%%%%%

% Make Layout UCL compliant

\setlrmarginsandblock{40mm}{20mm}{*}        % {spine margin}{edge margin}{ratio spine:edge}
\setulmarginsandblock{30mm}{*}{*}           % {upper margin}{lower margin}{ratio upper:lower}
\setheadfoot{\onelineskip}{3\onelineskip}   % Set the space for the header and footer
\setheaderspaces{*}{\onelineskip}{*}        % Set the spacing above/below the header
\checkandfixthelayout

% Add some extra space for long page numbers in toc
\setpnumwidth{3em}

% Modify the ruled style from memoir slightly and make the plain page numbers match
\makeheadrule{ruled}{\textwidth}{\normalrulethickness}
\makeevenhead{ruled}{\scshape\leftmark}{}{}
\makeoddhead{ruled}{}{}{\scshape\rightmark}
\makeevenfoot{plain}{\thepage}{}{}
\makeoddfoot{plain}{}{}{\thepage}


% Modify the marks for chapters and sections, and alter the marks for the headers of the bibliography etc. to make them small caps
\makepsmarks{ruled}{\nouppercaseheads
\createmark{chapter}{left}{shownumber}{}{:\,}
\createmark{section}{right}{shownumber}{}{:\,}
\createplainmark{toc}{both}{\contentsname}
\createplainmark{lof}{both}{\listfigurename}
\createplainmark{lot}{both}{\listtablename}
\createplainmark{lon}{both}{\listnotesname}
\createplainmark{bib}{both}{\bibname}
\createplainmark{index}{both}{\indexname}
\createplainmark{glossary}{both}{\glossaryname}}



%%%%%%%%%%%%%
%%% Draft %%%
%%%%%%%%%%%%%
\ifthenelse{\boolean{revision}}{
% Edit footers to include date of file generation
\newcommand{\draftString}{\textit{Draft: \today}}

\makeevenfoot{plain}{\thepage}{}{\draftString}
\makeoddfoot{plain}{\draftString}{}{\thepage}
\makeevenfoot{ruled}{\thepage}{}{\draftString}
\makeoddfoot{ruled}{\draftString}{}{\thepage}
}

\newcommand{\skipThis}[1]{\ignorespaces}

%%%%%%%%%%%%%%%%%%%%
%%% Environments %%%
%%%%%%%%%%%%%%%%%%%%

% Acknowledgements is similar to abstract enviroment...
\newenvironment{acknowledgements}
{\renewcommand{\abstractname}{Acknowledgements}\abstract}
{\endabstract}

% ... as is the dedication
\newenvironment{dedication}
{\renewcommand{\abstractname}{Dedication}\abstract}
{\endabstract}

\newenvironment{impact}
{\renewcommand{\abstractname}{Impact Statement}\abstract}
{\endabstract}


%%%%%%%%%%%%%%%%%%
%%% Appearence %%%
%%%%%%%%%%%%%%%%%%

% Include the short caption at the start of the caption, if provided
\makeatletter
\let\x@caption\caption % original \caption
\def\x@@caption[#1]#2{\x@caption[{#1}]{\textbf{#1} #2}} % with optional arg
\def\x@@@caption#1{\x@caption[{#1}]{#1}} % without optional arg
\def\caption{\@ifnextchar[\x@@caption\x@@@caption} % new \caption
\makeatother

% Define some useful commands for the chapter style
\makeatletter
\newlength\dlf@normtxtw
\setlength\dlf@normtxtw{\textwidth}
\newsavebox{\feline@chapter}

\newcommand\feline@chapter@marker[1][4cm]{%
\sbox\feline@chapter{%
\resizebox{!}{#1}{\fboxsep=1pt%
\colorbox{Highlight}{\color{white}\bfseries\thechapter}%
}}%
\rotatebox{90}{%
\resizebox{%
\heightof{\usebox{\feline@chapter}}+\depthof{\usebox{\feline@chapter}}}%
{!}{\scshape\@chapapp}}\quad%
\raisebox{\depthof{\usebox{\feline@chapter}}}{\usebox{\feline@chapter}}%
}

\newcommand\feline@chm[1][4cm]{%
\sbox\feline@chapter{\feline@chapter@marker[#1]}%
\makebox[0pt][l]{% aka \rlap
\makebox[1cm][r]{\usebox\feline@chapter}%
}}
\makeatother

% Prepare the chapter style(named after daleif1, upon which it was based)
\makeatletter
\makechapterstyle{Modifieddaleif1}{
\renewcommand\chapnamefont{\normalfont\Large\scshape\raggedleft\so}
\renewcommand\chaptitlefont{\normalfont\huge\bfseries\color{Highlight}}
\renewcommand\chapternamenum{}
\renewcommand\printchaptername{}
\renewcommand\printchapternum{\null\hfill\feline@chm[2.5cm]\par}
\renewcommand\afterchapternum{\par\vskip\midchapskip}
\renewcommand\printchaptertitle[1]{\chaptitlefont\raggedleft ##1\par}
}
\makeatother

\chapterstyle{Modifieddaleif1}

% Louise used a glossary - I didn't, so it has not been tested

% \newglossarystyle{listLB}{
% 
% \renewcommand*{\glossentry}[2]{%
% \item[\glsentryitem{##1}%
% \glstarget{##1}{\glossentryname{##1}}]
% \glossentrydesc{##1}\glspostdescription\\ \glsentryuseri{##1}}
% 
% % Sub-entries continue on the same line:
% \renewcommand*{\subglossentry}[3]{%
% \glssubentryitem{##2}%
% \glstarget{##2}{\strut}\space
% \glossentrydesc{##2}\glspostdescription\space ##3.}
% 
% % Add vertical space between groups:
% \renewcommand*{\glsgroupskip}{\ifglsnogroupskip\else\indexspace\fi}%
% }



%%%%%%%%%%%%%%%%%%%%
%%% Bibliography %%%
%%%%%%%%%%%%%%%%%%%%

\usepackage[authordate,backend=biber,uniquename=false]{biblatex-chicago}

%Fix so that TeXnicCenter recognises which bib file has been loaded						
\renewcommand{\bibliography}[1]{\addbibresource{#1.bib}}

\AtEveryBibitem{\clearfield{note}}
\AtEveryBibitem{\clearlist{language}}
\AtEveryBibitem{\clearfield{month}\clearfield{day}}

\DeclareBibliographyAlias{software}{online}

%make urls brake more easily:

\setcounter{biburllcpenalty}{7000}
\setcounter{biburlucpenalty}{8000}

%% Make the bibliography single spaced, and allow form more stretch. 
\AtBeginBibliography{\SingleSpacing \emergencystretch 3em}
		
%citeauthor with hyperlink
\DeclareCiteCommand{\calink}
  {\boolfalse{citetracker}%
   \boolfalse{pagetracker}%
   \usebibmacro{prenote}}
  {\ifciteindex
     {\indexnames{labelname}}
     {}%
   \printtext[bibhyperref]{\printnames{labelname}}}
  {\multicitedelim}
  {\usebibmacro{postnote}}

%citeyear with hyperlink
\DeclareCiteCommand{\cyl}
  {\boolfalse{citetracker}%
   \boolfalse{pagetracker}%
   \usebibmacro{prenote}}
  {\printtext[bibhyperref]{\printfield{year}}}
  {\multicitedelim}
  {\usebibmacro{postnote}}
		
%%% Some shortcuts for bibliography terms

\newcommand{\ca}{\citeauthor}
\newcommand{\ci}{\autocite}
\newcommand{\tc}{\textcite}

%Commacite
\newcommand{\cc}[1]{\calink{#1} \cyl{#1}}

%%% Shortcut for quotations

\newcommand{\bq}{\blockquote}
\newcommand{\te}{\textelp}
\newcommand{\tq}{\textquote}

%Make quotations using \bq and \begin{displayquote} italic and with british quotation marks.
\DeclareQuoteStyle[british]{english}
  {\itshape\textquoteleft}
  {\textquoteright}
  [0.05em]
  {\textquotedblleft}
  {\textquotedblright}
  

\renewcommand{\mkcitation}[1]{\ifblockquote{\sourceatright{(#1)}}{ (#1)}}

\renewcommand{\mkblockquote}[4]{\openautoquote#1#2\closeautoquote#3#4}

\SetBlockEnvironment{quotation}

\renewcommand{\mkbegdispquote}[2]{\openautoquote}
\renewcommand{\mkenddispquote}[2]{\closeautoquote#1#2}


% Cleveref stuff:

\usepackage[nameinlink,capitalize,noabbrev]{cleveref}   % Helpful package for writing figure names etc.

%Memoir subfigure / subtable setup
\newsubfloat{table}
\newsubfloat{figure}

% Define table columns  that allow for newlines:
\newcolumntype{L}[1]{>{\raggedright\let\newline\\\arraybackslash\hspace{0pt}}m{#1}}
\newcolumntype{E}[1]{>{\centering\let\newline\\\arraybackslash\hspace{0pt}}m{#1}}
\newcolumntype{R}[1]{>{\raggedleft\let\newline\\\arraybackslash\hspace{0pt}}m{#1}}

%Make \cref{app:diagrams} resolve to "Appendix A"
\crefname{appsec}{appendix}{Appendices}

% Set up a command for placeholder figures
\newcommand{\placeholder}[1]{\fbox{{\color{White}\rule{0.9\textwidth}{#1\textheight}}}}

